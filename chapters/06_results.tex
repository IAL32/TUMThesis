\chapter{Results}\label{chapter:results}

In this thesis, we are not looking to obtain the best possible combination of hyperparameters for training loss or model accuracy.
Instead, we want to observe the effects on training with Hivemind when tuning common hyperparameters such as batch size and learning rate and Hivemind hyperparameters such as the TBS.
In this thesis, we analyze the performance and limits of training using Hivemind rather than looking for the best model.

\section{Baseline runs}

We begin this chapter by showing the results that we have obtained with the baseline runs.
As mentioned previously in \autoref{chapter:setup}, all baseline experiments are executed on machines with the same configuration, and the total number of samples processed is always the same.
\autoref{fig:baseline-runtimes} shows the average runtimes for baseline runs in minutes.

\begin{figure}[h]
    \centering
    \foreach \gas in {1, 2}
        {
            \begin{subfigure}[b]{0.475\textwidth}
                \centering
                \caption{}
                \includegraphics[width=\textwidth]{./figures/06_barplot-runtime_gas-\gas_baseline-16vCPUs-GAS-1.pdf}
            \end{subfigure}%
            \hfill
        }
    \caption{Average runtimes of baseline experiments in minutes. Runs are aggregated across LR, with the standard deviation amongst reruns as the black bar.}
    \label{fig:baseline-runtimes}
\end{figure}

\begin{figure}[h]
    \centering
    \foreach \gas in {1, 2}
        {
            \begin{subfigure}[b]{\textwidth}
                \centering
                \caption{}
                \includegraphics[width=\textwidth]{./figures/06_barplot-losses_gas-\gas_baseline-16vCPUs-GAS-1.pdf}
            \end{subfigure}%
            \hfill
        }
    \caption{Loss achieved by baseline runs, averaged across re-runs.}
    \label{fig:baseline-losses}
\end{figure}

Baseline runs do not run distributed algorithms, all Hivemind features are switched off and machines do not communicate with each other.
However, \autoref{fig:net-recv_baseline} shows that there is some network activity.
On average, every machine receives a constant 1.5 MB/s of data on its network.
This may be due to several factors, such as KVM management data, OpenNebula pings, and CEPH data being read.

In the Setup section, we also introduced our monitoring tool of choice \textit{wandb}.
Because this is an online monitoring tool, some data about our runs is periodically sent to the Weights and Biases server for storage and visualizations.

In \autoref{fig:net-sent_baseline}, which shows the bandwidth used for send operations across all baseline runs, we can observe the bandwidth in MB/s used for each run.
On average, this is roughly 0.02 MB/s on every run, a value that can be mostly attributed to \textit{wandb} and other background monitoring operations such as OpenNebula.

In future sections, we will always account for these effects when performing comparisons with baseline runs.

\begin{figure}[h]
    \centering
    \begin{subfigure}[b]{0.475 \textwidth}
        \centering
        \caption{Network bandwidth sent in MB/s for baseline runs. Values above 0.07 are hidden. Runs are aggregated across LR.}
        \label{fig:net-sent_baseline}
        \includegraphics[width=\textwidth]{./figures/06_net-recv_baseline-16vCPUs-GAS-1.pdf}
    \end{subfigure}%
    \hfill
    \centering
    \begin{subfigure}[b]{0.475 \textwidth}
        \centering
        \caption{Network bandwidth received in MB/s for baseline runs. Values above 5 are hidden. Runs are aggregated across LR.}
        \label{fig:net-recv_baseline}
        \includegraphics[width=\textwidth]{./figures/06_net-sent_baseline-16vCPUs-GAS-1.pdf}
    \end{subfigure}%
    \hfill
    \caption{Network bandwidth sent and received in MB/s for baseline runs. Runs are aggregated across LR.}
\end{figure}

\autoref{fig:baseline-times-stacked} shows the average times for data load, forward pass, backward pass and optimization step across batch sizes in baseline runs for both GAS=1 and GAS=2.
As we might expect, the time it takes for a single step to complete is linearly dependent on the batch size.
The learning rate (LR) does not affect the time it takes for each step to complete, so we aggregated the runs for each batch size.
By contrast, the number of gradient accumulation steps (GAS) seems to shave off some time for every batch size, although the total runtimes in \autoref{fig:baseline-runtimes} do not seem to reflect this improvement.
Throughout this chapter, we will keep showing GAS runs separately, as it still might affect some other aspects of training.

\begin{figure}[h]
    \centering
    \includegraphics[width=\textwidth]{./figures/06_barplot-times_baseline-16vCPUs-GAS-1.pdf}
    \caption{
        Average times of step data load (small circles), forward pass (backward slash), backward pass (forward slash) and optimization step (stars) baseline experiments in seconds.
        Runs are further aggregated across LR and the standard error amongst runs is shown with black bars.
    }
    \label{fig:baseline-times-stacked}
\end{figure}


\section{Focus on effects of batch size, learning rate and target Batch Size}\label{sec:focus-effect-bs-lr-tbs}

Batch size and learning rate are some of the most fundamental hyperparameters to tune when training a neural network to obtain good training results.
Tuning the learning rate should not impact training performance directly, but it can help to better understand how to tune it for different settings combinations while using Hivemind.
As specified previously in \autoref{chapter:setup}, the reference optimizer algorithm is the stochastic gradient descent (SGD), which is wrapped around the \texttt{hivemind.Optimizer} class.

The batch size determines how many samples are being processed in a training loop.
In Hivemind, this has the consequence of reaching the TBS in fewer steps, but not necessarily in less time.

\autoref{fig:runtime-decrease_2-peers-8vCPUs} shows the runtimes for Hivemind experiments with 2 peers and 8vCPUs per peer compared to the baseline runs.
Every run shows a substantial decrease in runtime, with $BS=32$ having an average decrease of circa 20\%, $BS=64$ of circa 30\% and close to 40\% for $BS=128$.
But can we just expect such a high increase in performance for free when turning on Hivemind?
There are two important factors to take into consideration before making a such claim.

\begin{enumerate}
    \item Data loading in the baseline runs takes 1/3 of the total time per step as shown in \autoref{fig:baseline-times-stacked}.
          Parallelizing data loading indeed speeds up the overall runtime for each run.
          With further experimentation that is outside the scope of this thesis, it might be possible to reduce the data loading step with local parallelization techniques and faster storage.
          Reducing the data loading step might help rule out the possibility that we only see runtime improvements because of the effects of loading more data in parallel.
    \item The results in \autoref{fig:loss-increase_2-peers-8vCPUs} shows the hidden impact on loss of using Hivemind.
          Nearly all experiments are not able to reach the minimum loss set by the respective baseline runs.
          Some argue \cite{DBLP:journals/corr/abs-1708-03888} that it is possible to reach the same model accuracy just by training longer.
          Others \cite{DBLP:journals/corr/KeskarMNST16} argue that longer training with larger batch sizes might lead to overall worse generalization capabilities for the model.
          Proving the effects on accuracy and model generalization is beyond the scope of this thesis.
\end{enumerate}

\begin{figure}[h]
    \centering
    % temporary
    \foreach \gas in {1, 2}
        {
            \foreach \lu in {True, False}
                {
                    \begin{subfigure}[b]{0.24 \textwidth}
                        \caption{}
                        \includegraphics[width=\textwidth]{./figures/06_barplot-runtime_gas-\gas_lu-\lu_2-peers-8vCPUs.pdf}
                    \end{subfigure}%
                    \hfill
                }
        }
    \caption{Runtime decrease in percent for Hivemind runs with 2 peers and 8vCPUs relative to baseline runs. Higher is better. Runs are aggregated across LR and the standard error amongst runs is shown with black bars.}
    \label{fig:runtime-decrease_2-peers-8vCPUs}
\end{figure}

\begin{figure}[h]
    \centering
    % temporary
    \foreach \gas in {1, 2}
        {
            \foreach \lu in {True, False}
                {
                    \begin{subfigure}[b]{0.475\textwidth}
                        \centering
                        \caption{}
                        \includegraphics[width=\textwidth]{./figures/06_barplot-loss_gas-\gas_lu-\lu_2-peers-8vCPUs.pdf}
                    \end{subfigure}
                    \hfill
                }
        }
    \caption{Loss increase in percent for Hivemind runs with 2 peers and 8vCPUs relative to baseline runs. Higher is worse.}
    \label{fig:loss-increase_2-peers-8vCPUs}
\end{figure}


\begin{figure}[h]
    \centering
            \foreach \lu in {True, False}
                {

                    \begin{subfigure}[b]{\textwidth}
                        \centering
                        \caption{}
                        \includegraphics[width=\textwidth]{./figures/06_barplot-times_gas-1_lu-\lu_2-peers-8vCPUs.pdf}
                    \end{subfigure}%
                    \hfill
                }
    \caption{
        Average times of data load (small circles), forward pass (backward slash), backward pass (forward slash) and optimization step (stars) baseline experiments in seconds.
        Runs are further aggregated across LR and the standard error amongst runs is shown with black bars (continues).
    }
    \label{fig:times-stacked_2-peers-8vCPUs}
\end{figure}

\begin{figure}[htb]\ContinuedFloat % continue previous figure
    \centering
            \foreach \lu in {True, False}
                {

                    \begin{subfigure}[b]{\textwidth}
                        \centering
                        \caption{}
                        \includegraphics[width=\textwidth]{./figures/06_barplot-times_gas-2_lu-\lu_2-peers-8vCPUs.pdf}
                    \end{subfigure}%
                    \hfill
                }
    \caption*{Figure~\ref{fig:times-stacked_2-peers-8vCPUs}:~
        Average times of data load (small circles), forward pass (backward slash), backward pass (forward slash) and optimization step (stars) baseline experiments in seconds.
        Runs are further aggregated across LR and the standard error amongst runs is shown with black bars.
    }
\end{figure}

Depending on the optimizer used for training a neural network model, the number of parameters can become huge.
When performing an optimizer state averaging state, sending a high amount of parameters can lead to high communication overhead, and thus, reduced performance.
We can see this effect in \autoref{fig:net-recv-sys-bandwidth-mbs_2-peers-8vCPUs} and \autoref{fig:net-sent-sys-bandwidth-mbs_2-peers-8vCPUs}.
As the batch size increases, nodes send and receive more data, increasing bandwidth utilization.

\begin{figure}[h]
    \centering
    \foreach \gas in {1, 2}
        {
            \foreach \lu in {True, False}
                {
                    \begin{subfigure}[b]{0.24\linewidth}
                        \centering
                        \caption{}
                        \includegraphics[width=\textwidth]{./figures/06_net-recv-sys-bandwidth-mbs_gas-\gas_lu-\lu_2-peers-8vCPUs.pdf}
                    \end{subfigure}%
                    \hfill
                }
        }
    \caption{Network received for Hivemind runs with 2 peers and 8vCPUs. Values $>=20$ MB/s are hidden and runs are aggregated across LR.}
    \label{fig:net-recv-sys-bandwidth-mbs_2-peers-8vCPUs}
\end{figure}

\begin{figure}[h]
    \centering
    \foreach \gas in {1, 2}
        {
            \foreach \lu in {True, False}
                {
                    \begin{subfigure}[b]{0.24\linewidth}
                        \centering
                        \caption{}
                        \includegraphics[width=\textwidth]{./figures/06_net-sent-sys-bandwidth-mbs_gas-\gas_lu-\lu_2-peers-8vCPUs.pdf}
                    \end{subfigure}%
                    \hfill
                }
        }
    \caption{Network sent for Hivemind runs with 2 peers and 8vCPUs. Values $>=20$ MB/s are hidden and runs are aggregated across LR.}
    \label{fig:net-sent-sys-bandwidth-mbs_2-peers-8vCPUs}
\end{figure}

Considerations of training with Hivemind for the TBS, BS and LR hyperparameters:

\begin{itemize}
    \item With the same amount of computational power overall, training with Hivemind might need more time to reach the loss compared to the baseline runs.
    \item Having access to less powerful hardware still allows training peers to be helpful, at the cost of training for longer.
    \item Increasing the frequency of averaging does not make up for a bad selection of optimization hyperparameters such as the batch size and learning rate.
    \item However being able to perform averaging steps more frequently can help to reduce the loss gap with the baseline runs.
\end{itemize}


\section{Focus on effects of gradient accumulation}

Gradient accumulation allows the simulation of bigger batches within a single node by accumulating gradients every time the backpropagation step is performed.
After GAS steps, the optimizer step is performed and the gradients are finally applied to the trained model.
In theory, reducing the frequency of executing the optimizer step should also reduce the total time spent applying the gradients to the mode.
In practice, for small models like ResNet18, this doesn't make a discernible difference as shown in \autoref{fig:times-stacked_2-peers-8vCPUs}, where the optimizer step takes 0.05 seconds on average.

For loss, the scenario is quite different.
In \autoref{fig:loss-increase_2-peers-8vCPUs} we can see the loss increase with respect to the baseline runs in four different configurations:
\begin{itemize}
    \item GAS=1, LU=True; 
    \item GAS=1, LU=False;
    \item GAS=2, LU=True;
    \item GAS=2, LU=False;
\end{itemize}

With both LU=True and LU=False, we can notice a better loss with GAS=2 by 5-10\% compared to GAS=1 for experiments with high lower LR.
As LR increases, the gap between GAS=1 and GAS=2 closes, with the gap getting even closer for smaller TBS values.

Finally, the impact on network utilization using our configuration is minimal.
Possibly for scenarios with more traffic, high values of GAS may help reduce the number of times that the Hivemind optimizer is called, reducing step time.

Evaluating the benefits of using gradient accumulation and averaging, we can say the following when training ResNet18 on Imagenet with Hivemind:
\begin{itemize}
    \item the smaller the TBS, the less the difference between GAS=1 and GAS=2 matters. It remains an open question whether this statement holds for higher values of GAS.
    \item for high values of LR, GAS does not seem to affect training as much as for low values of LR;
\end{itemize}

\section{Focus on effects of local updates}

\section{Focus on effects of the number of peers and vCPUs per peer}

Institutions and companies may have more than two machines at their disposal to perform distributed training.
So far, we have explored the effects on Hivemind of specific settings such as TBS, BS LR, GAS and LU.
Adding more nodes to a distributed training setting can lead to bottlenecks, especially when using a client-server approach \cite{Atre_2021, 8886576}.
In this section, we answer the following research question: what are the effects of scaling up the number of machines when using Hivemind?

\begin{figure}[ht]
    \centering
    % temporary
    \foreach \gas in {1, 2}
        {
            \begin{subfigure}[t]{0.35 \textwidth}
                \caption{}
                \includegraphics[width=\textwidth]{./figures/06_barplot-runtime_gas-\gas_scale-nop.pdf}
            \end{subfigure}
        }
    \caption{Runtime decrease in percent for Hivemind runs with 2, 4, 8, 16 peers and 8, 4, 2, 1 vCPUs respectively relative to baseline runs. Higher is better. Runs are aggregated across LR and the standard error amongst runs is shown with black bars.}
    \label{fig:runtime-decrease_scale-nop}
\end{figure}

The frequency at which peers average their model state is directly proportional to the number of peers, the throughput per second of each peer and the TBS.
In turn, the throughput per second is affected by several factors such as the BS, computational power of the node and wait times for I/O operations.

It might be difficult to isolate the effects of introducing more nodes from scaling the target batch size.
Thus, we decided to fix the target batch size to 1250 for this set of experiments and alter TBS, BS, LR, GAS and LU.

\begin{figure}[ht]
    \centering
    \foreach \gas in {1}
        {
            \foreach \lu in {True, False}
                {
                    \begin{subfigure}[t]{0.45\linewidth}
                        \centering
                        \caption{}
                        \includegraphics[width=\textwidth]{./figures/06_barplot-loss_gas-\gas_lu-\lu_scale-nop.pdf}
                    \end{subfigure}
                }
        }
    \caption{GAS = 1, accuracy decrease in percent for Hivemind runs with 2, 4, 8, 16 peers and 8, 4, 2, 1 vCPUs respectively relative to baseline runs. Higher is worse.}
    \label{fig:loss-increase_scale-nop}
\end{figure}

\begin{figure}[ht]
    \centering
    \foreach \gas in {2}
        {
            \foreach \lu in {True, False}
                {
                    \begin{subfigure}[t]{0.45\linewidth}
                        \centering
                        \caption{}
                        \includegraphics[width=\textwidth]{./figures/06_barplot-loss_gas-\gas_lu-\lu_scale-nop.pdf}
                    \end{subfigure}
                }
        }
    \caption{GAS = 2, accuracy decrease in percent for Hivemind runs with 2, 4, 8, 16 peers and 8, 4, 2, 1 vCPUs respectively relative to baseline runs. Higher is worse.}
\end{figure}

As we might expect, \autoref{fig:runtime-decrease_scale-nop} shows that increasing the number of peers dramatically decreases runtime.
The highest jump in runtime performance is between using one single peer (Hivemind disabled) and using two peers (Hivemind enabled).
Introducing four peers also cuts down runtime by around 50\% compared to using two peers across all experiments.
However, this effect does not appear to be linear.
The benefits of including more peers only increase by 10-15\% for eight peers and 4-6\% for sixteen peers.
If we take into consideration the decreased accuracy performance, there seems to be a sweet spot in terms of reducing the total runtime and an acceptable reduction in accuracy performance.
Using four peers seems to be the optimal number of peers when training with Hivemind on our configuration to obtain the maximum reduction of runtime without having a significant hit in terms of accuracy.
It remains an open question whether training these runs for longer would yield the same accuracy as the baseline runs but in less time overall.
Further experimentation may also show that increasing the TBS, and thus reducing the averaging frequency amongst peers, can be beneficial in runs where TBS is quickly reached.

\autoref{fig:loss-increase_scale-nop} shows that GAS and LU settings seem to generally have a similar effect compared to Hivemind runs with 2 peers and 8vCPUs presented in \autoref{sec:focus-effect-bs-lr-tbs}.
The graph also shows us a decrease in performance as we increase the number of peers, especially for experiments that have reached a higher accuracy.
In general, we noticed that compared to baseline experiments with bad performance, the accuracy does not change too much when using Hivemind.

\autoref{fig:times-stacked_scale-nop} shows the average time taken for each step in every different combination for the experiments changing NoP.
The increase in time taken for each operation is consistent with what we would expect: halving the number of computational power results in double the time taken per operation.
This is connected with the reduction in runtime increase benefits shown before.
As we half the computational power per peer, slower operations such as data loading, forward and backward pass stack up, eventually leading to slower runtimes.
It remains an open question however to see if increasing the number of peers and increasing the available computational power will solve these issues and by how much.

The network bandwidth utilization for different peer configurations is shown in \autoref{fig:net-recv-sys-bandwidth-mbs_scale-nop} and \autoref{fig:net-sent-sys-bandwidth-mbs_scale-nop}, and yield interesting results.
As we increase the number of peers in a training session, peers communicate more often, which can also be seen as these "bulbs" in the violin plots.
This is unsurprising, for two reasons:
first, the time to reach the fixed TBS of 1250 gets shorter as we add more peers, thus, the frequency at which peers communicate increases with the number of peers;
second, with more peers to average with, there is more data to exchange in terms of pings, synchronization messages and such.
We can see that with sixteen peers, the sent bandwidth utilization is almost exclusively around 5MB/s.
Nevertheless, even with sixteen peers, we did not see any significant CPU bottleneck caused by high network communication.

Finally, we make different observations for GAS and LU values as we presented in \autoref{sec:focus-local-updates} and \autoref{sec:focus-gradient-acc}.
Enabling LU for a higher number of peers seems to be very penalizing, especially with NoP=16.
Disabling LU instead yields more consistent accuracies in every experiment but seems to perform best for large values of LR.
Increasing GAS appears to worsen the situation when paired with LU=True, except for very low values of LR.
On the other hand, increasing GAS to 2 while disabling local updates seems to be a bit better, but not as consistently as decreasing TBS in \autoref{sec:focus-effect-bs-lr-tbs}.

Summarizing the findings, we can say the following for our setup:
\begin{itemize}
    \item Increasing the number of peers while maintaining the same computational power can reduce the total runtime by at least 30\%.
    \item However runtime reduction is not linear compared to the number of peers.
          The effects of reducing the data load times by using faster storage are still an open question.
    \item With local updates enabled, increasing the number of peers seems to have a worse effect on training accuracy.
          The effects of other values of TBS for a different number of peers is still an open question.
    \item Introducing more peers leads to more bandwidth usage as each peer exchanges more data with other averaging partners.
          This effect can become much larger for larger models and a much larger number of nodes.
\end{itemize}%

\begin{figure}[H]
    \centering
    \foreach \gas in {1, 2}
        {
            \begin{subfigure}[t]{0.34\textwidth}
                \centering
                \caption{}
                \includegraphics[width=\textwidth]{./figures/06_barplot-times_gas-\gas_scale-nop.pdf}
            \end{subfigure}%
        }
    \caption{
        Average times of data load (red), forward pass (green), backward pass (orange) and optimization step (blue) for Hivemind runs with 2, 4, 8, 16 peers and 8, 4, 2, 1 vCPUs respectively in seconds.
        Runs are further aggregated across LR.
    }
    \label{fig:times-stacked_scale-nop}
\end{figure}%


\begin{figure}[H]
    \centering
    \foreach \gas in {1, 2}
        {
            \begin{subfigure}[t]{0.30\linewidth}
                \centering
                \caption{}
                \includegraphics[width=\textwidth]{./figures/06_net-recv-sys-bandwidth-mbs_gas-\gas_scale-nop.pdf}
            \end{subfigure}%
        }
    \caption{Network received for Hivemind runs with 2, 4, 8, 16 peers and 8, 4, 2, 1 vCPUs respectively. Values $\geq 20$ MB/s are hidden and runs are aggregated across LR.}
    \label{fig:net-recv-sys-bandwidth-mbs_scale-nop}
\end{figure}%
\begin{figure}[H]
    \centering
    \foreach \gas in {1, 2}
        {
            \begin{subfigure}[t]{0.30\linewidth}
                \centering
                \caption{}
                \includegraphics[width=\textwidth]{./figures/06_net-sent-sys-bandwidth-mbs_gas-\gas_scale-nop.pdf}
            \end{subfigure}%
        }
    \caption{Network sent for Hivemind runs with 2, 4, 8, 16 peers and 8, 4, 2, 1 vCPUs respectively. Values $\geq 20$ MB/s are hidden and runs are aggregated across LR.}
    \label{fig:net-sent-sys-bandwidth-mbs_scale-nop}
\end{figure}

