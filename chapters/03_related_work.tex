\chapter{RELATED WORK}\label{chapter:related-work}
In this chapter, we provide the necessary background to understand this thesis and discuss related work.

\section{DATA PARALLELISM}
Throughout the years, datasets and machine learning models have grown in size, requiring distributed techniques to train them efficiently.
Data parallelism is the most common type of distributed learning and has been implemented on many common frameworks such as PyTorch \footnote{\href{https://pytorch.org/}{https://pytorch.org/}} and TensorFlow\footnote{\href{https://tensorflow.org/}{https://tensorflow.org/}} through a high-level API.
It has been known however that operations such as updating and distributing large neural network models can run into network bottlenecks \cite{10.48550/arxiv.1705.08741, DBLP:journals/corr/abs-2003-11316, 10.5555/2999134.2999271, DBLP:journals/corr/abs-1811-03600}.
Because of this, operations such as pruning and quantization are common to reduce the impact of communication in a distributed training setting \cite{10.48550/arxiv.2003.03033, 10.48550/arxiv.1510.00149}.
Despite these efforts, increasing the number of nodes and model size can still cause network bottlenecks.
Less known are the effects of preprocessing bottlenecks, which can severely affect the performance of models before training even starts \cite{isenko2022bottleneck, 10.1145/3448016.3457566}.
In this thesis, we show a general approach for studying and evaluating the effects of bottlenecks in data parallelism using Hivemind.

\section{LARGE BATCH TRAINING}
Training neural networks using large or very large batches \cite{DBLP:journals/corr/KeskarMNST16, 10.48550/arxiv.1705.08741} has gained traction amongst researchers.
Using small batch sizes is generally preferable when using minibatch SGD algorithms as it allows escaping local minima within a few iterations.
Increasing the number of minibatches also decreases the update frequency, making training faster.
However, this also leads to the possibility of getting stuck on local minima due to less noisy updates and thus low training performance overall.

The attractiveness of using large batches is to leverage the parallelism properties of distributed optimization algorithms.
\citeauthor{goyal2017accurate} have managed to train Imagenet in around 1 hour using a batch size of 8192 \cite{goyal2017accurate}.
This was possible as they showed that the decrease in performance when increasing the minibatch size is highly non-linear, staying constant for values up until 8192 and increasing rapidly afterward.
As training can still diverge in the first parts of training, they introduce a warmup period where a single node starts training, after which more nodes can join.

In their experiments, \citeauthor{ryabinin2020learning} show that it is possible to perform large batch training in a volunteer computing setting using Hivemind \cite{ryabinin2020learning, DBLP:journals/corr/abs-2106-10207}.
Hivemind allows setting different strategies for when parameters and model states are averaged, effectively simulating larger batches across participating peers.
However, there is a limited amount of experiments showcasing different Hivemind configurations in a controlled environment.

\section{VOLUNTEER COMPUTING}
Researchers today are dealing with hard and complex problems such as particle simulation and protein folding.
Because of their nature, these problems require immense amounts of computational power, which are not always available to single institutions.
Supercomputers can help tackle these problems, but access to them is limited.
In the early days of distributed computing, it was found that researchers could leverage the power of volunteers who donate their idle devices to help solve their problems.
This paradigm is denominated \textit{volunteer computing} (VC).

Standard data parallelism distributed algorithms are made with node reliability in mind, and a node failure may simply cause the whole training to fail.
However, in VC, a node dropping off mid-training is an expected case.
For these reasons, few works have successfully implemented VC for deep learning training.
Most VC works use a server-client infrastructure, where the server is hosted and managed by the runners of the experiments and the clients are the volunteers \cite{Atre_2021, 8886576}.
The server in this case becomes a single point of failure and possible bottleneck, which can become an issue if an elevated number of volunteers joins the training.

The Hivemind framework \cite{ryabinin2021mosphit} uses a modified version of minibatch SGD \cite{ryabinin2021mosphit} that considers node failure.
Furthermore, the authors implemented DeDLOC, a dedicated training algorithm \cite{DBLP:journals/corr/abs-2106-10207} that removes the single node of failure bottleneck.
Hivemind has been proven successful in an experiment training a modified version of DALL-E \cite{ramesh2021zero} with 40 volunteers over two months.
