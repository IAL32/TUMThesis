\chapter{Related Work}\label{chapter:related-work}
Data parallelism has been leveraged to improve the performance of machine learning systems \cite{alexnet2012}.
Several common frameworks implement data parallelism such as PyTorch \footnote{\href{https://pytorch.org/}{https://pytorch.org/}} and TensorFlow\footnote{\href{https://tensorflow.org/}{https://tensorflow.org/}} through a high-level API.
There are already some works that analyze the effects of data parallelism \cite{DBLP:journals/corr/abs-2003-11316,DBLP:journals/corr/abs-1811-03600}.
Throughout our experiments, we have found similar results in terms of data parallelism limits.
We show that increasing the number of machines and the frequency of communication between them when training with Hivemind can reach a point of diminishing returns.

Recently, training neural networks using large or very large batches \cite{DBLP:journals/corr/KeskarMNST16, 10.48550/arxiv.1705.08741} has gained traction amongst researchers.
Hivemind allows setting different strategies for when parameters and model states are averaged, effectively simulating larger batches.
However, in their original paper \cite{ryabinin2020learning} the authors do not go into the details about what is the rationale for their specific algorithm implementation, and its effect on training.
Instead, they focus on the Mixture-of-Experts algorithm implementation for Hivemind, which has not yet been tested in a real-life scenario because the software is yet to be completed.
In this thesis, we would like to provide a deeper insight into the effects of Hivemind when training with larger batches, tuning the local updates setting and implementing gradient accumulation to our training experiments.

It is challenging to find a good balance between creating experiments that cover a lot of parameters and practicality.
More often than not, it is not possible to explore all possible combinations and tune every knob of a system, due to budget or time restrictions.
As explained by \Citeauthor{DBLP:journals/corr/abs-1811-03600} in their analysis, budgets are also often ignored during the evaluation of novel techniques.
This can lead to a bias towards positive results, limiting the application of the lessons learned from each research paper.
In this thesis, we set an arbitrary budget of 320,000 samples per experiment to evaluate our metrics.
We further limit the number of hyperparameters to explore for both regular training and baseline experiments and Hivemind experiments.
In total, we produced 288 experiments that cover a small number of settings.
This allows us to focus on the effects of tuning single parameters rather than trying to achieve the best possible model.

