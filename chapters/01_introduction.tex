% !TeX root = ../main.tex
% Add the above to each chapter to make compiling the PDF easier in some editors.

\chapter{Introduction}\label{chapter:introduction}

% What are NNs?
% - Introduction to Neural Networks (4-5 pages + figures)
% - Introduction to training a NN (3-4 pages + figures)

% What is the Hivemind?
% - Introduction to distributed training (4-5 pages + figures)
% - Introduction to Hivemind (3-4 pages + figures)

% What is the goal of this thesis?
% - Introducction to bottleneck analysis (2-3 pages + figures)
% - What measurements are we taking into consideration? E.g. bandwidth, cpu, memory, ... (2 pages + figures)

% Setup
% - Introduction to our systems architecture (2 pages + figures)

% Experiments
% - Introduction to first set of experiments (2 pages + table)
% (compare training ResNet18 on Imagenet with 10k steps w.r.t distributed Hivemind setup)
% - What are we looking for? (2 pages)
% (is training with multiple peers really faster? And by how much? Given the same amount of resources and steps, is it better to train on a single node or with Hivemind?)
% (what about distributed storage? Is there a limit to how many peers can access the same distributed storage such as CEPH?)
% [optional] (can we make distributed storage smarter? E.g. use Hivemind's DHT to smartly assign batches to each peer)
% - What have we found? (1-2 pages)
% (Training with Hivemind is faster by Nx times, scaling linearly/exponentially/logarithmically)
% - Introduction to second set of experiments [...]
% [Ideally]
% Train ResNet18 (or other) on as many resources as we can using Hivemind, see how long it takes to reach SOTA

% Conclusions
% - Are there any patterns across our experiments?
% - When using Hivemind, what do we have to look out for?
% - Are there any obvious issues and caveats that make training with Hivemind better than single-node training?
% - Can we make any suggestions and improvements as to how to train distributed CV models using Hivemind?


Neural networks (NNs) have been one of the main focuses of research in the past decade.
They were first conceptualized back in 

\section{Section}


\subsection{Subsection}
