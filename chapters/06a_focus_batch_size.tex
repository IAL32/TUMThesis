
\section{Focus on effects of batch size, learning rate and target Batch Size}\label{sec:focus-effect-bs-lr-tbs}

Batch size and learning rate are some of the most fundamental hyperparameters to tune when training a neural network to obtain good training results.
Tuning the learning rate should not impact training performance directly, but it can help to better understand how to tune it for different settings combinations while using Hivemind.
As specified previously in \autoref{chapter:setup}, the reference optimizer algorithm is the stochastic gradient descent (SGD), which is wrapped around the \texttt{hivemind.Optimizer} class.

The batch size determines how many samples are being processed in a training loop.
In Hivemind, this has the consequence of reaching the TBS in fewer steps, but not necessarily in less time.

\autoref{fig:runtime-decrease_2-peers-8vCPUs} shows the runtimes for Hivemind experiments with 2 peers and 8vCPUs per peer compared to the baseline runs.
Every run shows a substantial decrease in runtime, with $BS=32$ having an average decrease of circa 20\%, $BS=64$ of circa 30\% and close to 40\% for $BS=128$.
But can we just expect such a high increase in performance for free when turning on Hivemind?
There are two important factors to take into consideration before making a such claim.

\begin{enumerate}
    \item Data loading in the baseline runs takes 1/3 of the total time per step as shown in \autoref{fig:baseline-times-stacked}.
          Parallelizing data loading indeed speeds up the overall runtime for each run.
          With further experimentation that is outside the scope of this thesis, it might be possible to reduce the data loading step with local parallelization techniques and faster storage.
          Reducing the data loading step might help rule out the possibility that we only see runtime improvements because of the effects of loading more data in parallel.
    \item The results in \autoref{fig:loss-increase_2-peers-8vCPUs} shows the hidden impact on loss of using Hivemind.
          Nearly all experiments are not able to reach the minimum loss set by the respective baseline runs.
          Some experiments \cite{you2017scaling} have shown that large batch training can lead to divergence, and it is possible to reach the same model accuracy just by training longer.
          Others \cite{DBLP:journals/corr/KeskarMNST16} argue that longer training with larger batch sizes might lead to overall worse generalization capabilities for the model.
          Proving the effects on accuracy and model generalization is beyond the scope of this thesis.
\end{enumerate}

\begin{figure}[h]
    \centering
    % temporary
    \foreach \gas in {1, 2}
        {
            \begin{subfigure}[b]{0.475 \textwidth}
                \caption{}
                \includegraphics[width=\textwidth]{./figures/06_barplot-runtime_gas-\gas_2-peers-8vCPUs.pdf}
            \end{subfigure}%
            \hfill
        }
    \caption{Runtime decrease in percent for Hivemind runs with 2 peers and 8vCPUs relative to the baseline runs. Higher is better. Runs are aggregated across LR and the standard error amongst runs is shown with black bars.}
    \label{fig:runtime-decrease_2-peers-8vCPUs}
\end{figure}

\begin{figure}[h]
    \centering
    % temporary
    \foreach \gas in {1, 2}
        {
            \foreach \lu in {True, False}
                {
                    \begin{subfigure}[b]{0.475\textwidth}
                        \centering
                        \caption{}
                        \includegraphics[width=\textwidth]{./figures/06_barplot-loss_gas-\gas_lu-\lu_2-peers-8vCPUs.pdf}
                    \end{subfigure}
                    \hfill
                }
        }
    \caption{Loss increase in percent for Hivemind runs with 2 peers and 8vCPUs relative to the baseline runs. Higher is worse.}
    \label{fig:loss-increase_2-peers-8vCPUs}
\end{figure}


\begin{figure}[h]
    \centering
    \foreach \lu in {True, False}
        {

            \begin{subfigure}[b]{\textwidth}
                \centering
                \caption{}
                \includegraphics[width=\textwidth]{./figures/06_barplot-times_gas-1_lu-\lu_2-peers-8vCPUs.pdf}
            \end{subfigure}%
            \hfill
        }
    \caption{
        Average times of data load (small circles), forward pass (backward slash), backward pass (forward slash) and optimization step (stars) baseline experiments in seconds.
        Runs are further aggregated across LR and the standard error amongst runs is shown with black bars (continues).
    }
    \label{fig:times-stacked_2-peers-8vCPUs}
\end{figure}

\begin{figure}[htb]\ContinuedFloat % continue previous figure
    \centering
    \foreach \lu in {True, False}
        {

            \begin{subfigure}[b]{\textwidth}
                \centering
                \caption{}
                \includegraphics[width=\textwidth]{./figures/06_barplot-times_gas-2_lu-\lu_2-peers-8vCPUs.pdf}
            \end{subfigure}%
            \hfill
        }
    \caption*{Figure~\ref{fig:times-stacked_2-peers-8vCPUs}:~
        Average times of data load (small circles), forward pass (backward slash), backward pass (forward slash) and optimization step (stars) baseline experiments in seconds.
        Runs are further aggregated across LR and the standard error amongst runs is shown with black bars.
    }
\end{figure}

Depending on the optimizer used for training a neural network model, the number of parameters can become huge.
When performing an optimizer state averaging state, sending a high amount of parameters can lead to high communication overhead, and thus, reduced performance.
We can see this effect in \autoref{fig:net-recv-sys-bandwidth-mbs_2-peers-8vCPUs} and \autoref{fig:net-sent-sys-bandwidth-mbs_2-peers-8vCPUs}.
As the batch size increases, nodes send and receive more data, increasing bandwidth utilization.

\begin{figure}[h]
    \centering
    \foreach \gas in {1, 2}
        {
            \begin{subfigure}[b]{0.475\linewidth}
                \centering
                \caption{}
                \includegraphics[width=\textwidth]{./figures/06_net-recv-sys-bandwidth-mbs_gas-\gas_2-peers-8vCPUs.pdf}
            \end{subfigure}%
            \hfill
        }
    \caption{Network received for Hivemind runs with 2 peers and 8vCPUs. Values $\geq 10$ MB/s are hidden and runs are aggregated across LR.}
    \label{fig:net-recv-sys-bandwidth-mbs_2-peers-8vCPUs}
\end{figure}

\begin{figure}[h]
    \centering
    \foreach \gas in {1, 2}
        {
            \begin{subfigure}[b]{0.475\linewidth}
                \centering
                \caption{}
                \includegraphics[width=\textwidth]{./figures/06_net-sent-sys-bandwidth-mbs_gas-\gas_2-peers-8vCPUs.pdf}
            \end{subfigure}%
            \hfill
        }
    \caption{Network sent for Hivemind runs with 2 peers and 8vCPUs. Values $\geq 10$ MB/s are hidden and runs are aggregated across LR.}
    \label{fig:net-sent-sys-bandwidth-mbs_2-peers-8vCPUs}
\end{figure}

Considerations of training with Hivemind for the TBS, BS and LR hyperparameters:

\begin{itemize}
    \item With the same amount of computational power overall, training with Hivemind might need more time to reach the loss compared to the baseline runs.
    \item Having access to less powerful hardware still allows training peers to be helpful, at the cost of training for longer.
    \item Increasing the frequency of averaging does not make up for a bad selection of optimization hyperparameters such as the batch size and learning rate.
    \item However being able to perform averaging steps more frequently can help to reduce the loss gap with the baseline runs.
\end{itemize}
